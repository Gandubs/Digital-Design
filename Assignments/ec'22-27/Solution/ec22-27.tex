%\iffalse
\let\negmedspace\undefined
\let\negthickspace\undefined
\documentclass[journal,12pt,onecolumn]{IEEEtran}
\usepackage{cite}
\usepackage{amsmath,amssymb,amsfonts,amsthm}
\usepackage{algorithmic}
\usepackage{graphicx}
\usepackage{textcomp}
\usepackage{xcolor}
\usepackage{txfonts}
\usepackage{listings}
\usepackage{enumitem}
\usepackage{mathtools}
\usepackage{gensymb}
\usepackage{comment}
\usepackage[breaklinks=true]{hyperref}
\usepackage{tkz-euclide} 
\usepackage{listings}
\usepackage{gvv}    
\usepackage{enumitem}
\usepackage{amsmath}
\def\inputGnumericTable{}                                 
\usepackage[latin1]{inputenc}                                
\usepackage{color}                                            
\usepackage{array}                                            
\usepackage{longtable}                                       
\usepackage{calc}                                             
\usepackage{multirow}                                         
\usepackage{hhline}                                           
\usepackage{ifthen}                                           
\usepackage{lscape}
\usepackage{tabularx}
\usetikzlibrary{shapes, arrows, positioning}

\newtheorem{theorem}{Theorem}[section]
\newtheorem{problem}{Problem}
\newtheorem{proposition}{Proposition}[section]
\newtheorem{lemma}{Lemma}[section]
\newtheorem{corollary}[theorem]{Corollary}
\newtheorem{example}{Example}[section]
\newtheorem{definition}[problem]{Definition}
\newcommand{\BEQA}{\begin{eqnarray}}
\newcommand{\EEQA}{\end{eqnarray}}
\newcommand{\define}{\stackrel{\triangle}{=}}
\theoremstyle{remark}
\newtheorem{rem}{Remark}
\begin{document}
\bibliographystyle{IEEEtran}
\vspace{3cm}

\title{GATE:EC-27-2022}
\author{EE23BTECH11025 - Anantha Krishnan $^{}$% <-this % stops a space
}
\maketitle
\bigskip



\section{question}
Select the boolean function(s) equivalent to $x+yz$, where $x$,$y$ and $z$ are Boolean variables, and $+$ denotes logical OR operation.
\begin{enumerate}
    \item [(A)] $x+z+xy$
    \item [(B)] $\brak{x+y}\brak{x+z}$
    \item [(C)] $x+xy+yz$
    \item [(D)] $x+xz+xy$
\end{enumerate}

\section{Solution}
Simplifying each option to their simplest form:
\begin{enumerate}
    \item [(A)]
    \begin{align}
        x+z+xy &= x\brak{1+y} +z\\
        &=x+z
    \end{align}
    \item [(B)]
    \begin{align}
    \brak{x+y}\brak{x+z} &= x+xz+xy+yz\\
    &=x\brak{1+z}+xy+yz\\
    &=x\brak{1+y}+yz\\
    &=x+yz
        \end{align}
    \item [(C)]
    \begin{align}
        x+xy+yz &= x\brak{1+y}+yz\\
        &=x+yz
    \end{align}
    \item [(D)]
    \begin{align}
        x+xz+xy &= x\brak{1+z}+xy\\
        &=x\brak{1+y}\\
        &=x
    \end{align}
\end{enumerate}
Therefore option (B) and (C) are true.\\
The following is the implementation with a cpp code through esp-32 via Vaman.\\
\url{https://github.com/Gandubs/Digital-Design/blob/master/Assignments/ec'22-27/Codes/main.cpp}
 \end{document}
