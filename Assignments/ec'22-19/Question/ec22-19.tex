%\iffalse
\let\negmedspace\undefined
\let\negthickspace\undefined
\documentclass[journal,12pt,onecolumn]{IEEEtran}
\usepackage{cite}
\usepackage{amsmath,amssymb,amsfonts,amsthm}
\usepackage{algorithmic}
\usepackage{graphicx}
\usepackage{textcomp}
\usepackage{xcolor}
\usepackage{txfonts}
\usepackage{listings}
\usepackage{enumitem}
\usepackage{mathtools}
\usepackage{gensymb}
\usepackage{comment}
\usepackage[breaklinks=true]{hyperref}
\usepackage{tkz-euclide} 
\usepackage{listings}
\usepackage{gvv}    
\usepackage{enumitem}
\usepackage{amsmath}
\def\inputGnumericTable{}                                 
\usepackage[latin1]{inputenc}                                
\usepackage{color}                                            
\usepackage{array}                                            
\usepackage{longtable}                                       
\usepackage{calc}                                             
\usepackage{multirow}                                         
\usepackage{hhline}                                           
\usepackage{ifthen}                                           
\usepackage{lscape}
\usepackage{tabularx}

\newtheorem{theorem}{Theorem}[section]
\newtheorem{problem}{Problem}
\newtheorem{proposition}{Proposition}[section]
\newtheorem{lemma}{Lemma}[section]
\newtheorem{corollary}[theorem]{Corollary}
\newtheorem{example}{Example}[section]
\newtheorem{definition}[problem]{Definition}
\newcommand{\BEQA}{\begin{eqnarray}}
\newcommand{\EEQA}{\end{eqnarray}}
\newcommand{\define}{\stackrel{\triangle}{=}}
\theoremstyle{remark}
\newtheorem{rem}{Remark}
\begin{document}
\bibliographystyle{IEEEtran}
\vspace{3cm}

\title{GATE:IN-42-2023}
\author{Anantha Krishnan $^{}$% <-this % stops a space
}
\maketitle
\bigskip



\section{question}
Consider the $2$-bit multiplexer(MUX) shown in the figure. For OUTPUT to be the XOR of $C$ and $D$, the values of $A_0$,$A_1$,$A_2$ and $A_3$ are:
\begin{enumerate}
    \item  $A_0=0$, $A_1=0$, $A_2=1$, $A_3=1$
    \item  $A_0=1$, $A_1=0$, $A_2=1$, $A_3=0$
    \item  $A_0=0$, $A_1=1$, $A_2=1$, $A_3=0$
    \item  $A_0=1$, $A_1=1$, $A_2=0$, $A_3=0$
\end{enumerate}
\begin{tikzpicture}

    % MUX box
    \draw (2, 0) rectangle (6, 4);
    
    % Input lines and labels
    \draw[->] (1, 0.5) -- (2, 0.5);
    \draw[->] (1, 1.5) -- (2, 1.5);
    \draw[->] (1, 2.5) -- (2, 2.5);
    \draw[->] (1, 3.5) -- (2, 3.5);

    \node[left] at (0.8, 0.5) {$A_3$};
    \node[left] at (0.8, 1.5) {$A_2$};
    \node[left] at (0.8, 2.5) {$A_1$};
    \node[left] at (0.8, 3.5) {$A_0$};

    \node[left] at (0.2, 2) {INPUT};

    % Select lines and labels
    \draw[->] (3, -1) -- (3, 0);
    \draw[->] (5, -1) -- (5, 0);

    \node[below] at (3, 0.7) {$S_1$};
    \node[below] at (5, 0.7) {$S_0$};
    \node at (4, 0.7) {SELECT};

    \node[below] at (3, -0.9) {$C$};
    \node[below] at (5, -0.9) {$D$};

    % Output line and label
    \draw[->] (6, 2) -- (7, 2) node[below] {OUTPUT};

    % Labels inside MUX box
    \node at (2.2, 3.5) {0};
    \node at (2.2, 2.5) {1};
    \node at (2.2, 1.5) {2};
    \node at (2.2, 0.5) {3};

\end{tikzpicture}
 \end{document}
