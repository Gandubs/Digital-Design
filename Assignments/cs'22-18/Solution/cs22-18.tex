%\iffalse
\let\negmedspace\undefined
\let\negthickspace\undefined
\documentclass[journal,12pt,onecolumn]{IEEEtran}
\usepackage{cite}
\usepackage{amsmath,amssymb,amsfonts,amsthm}
\usepackage{algorithmic}
\usepackage{graphicx}
\usepackage{textcomp}
\usepackage{xcolor}
\usepackage{txfonts}
\usepackage{listings}
\usepackage{enumitem}
\usepackage{mathtools}
\usepackage{gensymb}
\usepackage{comment}
\usepackage[breaklinks=true]{hyperref}
\usepackage{tkz-euclide} 
\usepackage{listings}
\usepackage{gvv}    
\usepackage{enumitem}
\usepackage{amsmath}
\def\inputGnumericTable{}                                 
\usepackage[latin1]{inputenc}                                
\usepackage{color}                                            
\usepackage{array}                                            
\usepackage{longtable}                                       
\usepackage{calc}                                             
\usepackage{multirow}                                         
\usepackage{hhline}                                           
\usepackage{ifthen}                                           
\usepackage{lscape}
\usepackage{tabularx}

\newtheorem{theorem}{Theorem}[section]
\newtheorem{problem}{Problem}
\newtheorem{proposition}{Proposition}[section]
\newtheorem{lemma}{Lemma}[section]
\newtheorem{corollary}[theorem]{Corollary}
\newtheorem{example}{Example}[section]
\newtheorem{definition}[problem]{Definition}
\newcommand{\BEQA}{\begin{eqnarray}}
\newcommand{\EEQA}{\end{eqnarray}}
\newcommand{\define}{\stackrel{\triangle}{=}}
\theoremstyle{remark}
\newtheorem{rem}{Remark}
\begin{document}
\bibliographystyle{IEEEtran}
\vspace{3cm}

\title{GATE:CS22-18}
\author{Anantha Krishnan $^{}$% <-this % stops a space
}
\maketitle
\bigskip



\section{question}
Let R1 and R2 be two $4$-bit registers that store numbers in $2$'s complement form. For the operation R1 and R2, which one of the following values of R1 and R2 gives an arithmetic overflow?
\begin{enumerate}
    \item [(A)] R1 = 1011 and R2 = 1110
        \item [(B)] R1 = 1100 and R2 = 1010
    \item [(C)] R1 = 0011 and R2 = 0100
    \item [(D)] R1 = 1001 and R2 = 1111
\end{enumerate}
\section{Solution}
Converting each 2's complement to their binary equivalent, Adding up and converting to 2's complement form:

\begin{enumerate}
    \item [(A)]
    \begin{align}
R1 = -\brak{0101} \text{ and } R2=-\brak{0010}
\end{align}
\begin{align}
    R1 + R2 = 1001
\end{align}

    \item [(B)]
    \begin{align}
R1 = -\brak{0100} \text{ and } R2=-\brak{0110}
\end{align}
\begin{align}
    R1 + R2 = 0110
\end{align}

    \item [(C)]
    \begin{align}
R1 = +\brak{0011} \text{ and } R2=+\brak{0100}
\end{align}
\begin{align}
    R1 + R2 = 0111
\end{align}

    \item [(D)]
    \begin{align}
R1 = -\brak{0111} \text{ and } R2=-\brak{0001}
\end{align}
\begin{align}
    R1 + R2 = 1000
\end{align}
\end{enumerate}
In option (B) the signed bit of the sum changed from each of the both same signed numbers indicating an overflow.
Code for implementation through platformio
write link here
 \end{document}
