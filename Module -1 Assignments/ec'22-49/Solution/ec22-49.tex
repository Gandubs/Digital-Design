%\iffalse
\let\negmedspace\undefined
\let\negthickspace\undefined
\documentclass[journal,12pt,onecolumn]{IEEEtran}
\usepackage{cite}
\usepackage{amsmath,amssymb,amsfonts,amsthm}
\usepackage{algorithmic}
\usepackage{graphicx}
\usepackage{textcomp}
\usepackage{xcolor}
\usepackage{txfonts}
\usepackage{listings}
\usepackage{enumitem}
\usepackage{mathtools}
\usepackage{gensymb}
\usepackage{comment}
\usepackage[breaklinks=true]{hyperref}
\usepackage{tkz-euclide} 
\usepackage{listings}
\usepackage{gvv}    
\usepackage{enumitem}
\usepackage{amsmath}
\def\inputGnumericTable{}                                 
\usepackage[latin1]{inputenc}                                
\usepackage{color}                                            
\usepackage{array}                                            
\usepackage{longtable}                                       
\usepackage{calc}                                             
\usepackage{multirow}                                         
\usepackage{hhline}                                           
\usepackage{ifthen}                                           
\usepackage{lscape}
\usepackage{tabularx}
\usetikzlibrary{shapes, arrows, positioning}

\newtheorem{theorem}{Theorem}[section]
\newtheorem{problem}{Problem}
\newtheorem{proposition}{Proposition}[section]
\newtheorem{lemma}{Lemma}[section]
\newtheorem{corollary}[theorem]{Corollary}
\newtheorem{example}{Example}[section]
\newtheorem{definition}[problem]{Definition}
\newcommand{\BEQA}{\begin{eqnarray}}
\newcommand{\EEQA}{\end{eqnarray}}
\newcommand{\define}{\stackrel{\triangle}{=}}
\theoremstyle{remark}
\newtheorem{rem}{Remark}
\begin{document}
\bibliographystyle{IEEEtran}
\vspace{3cm}

\title{GATE:EC-49-2022}
\maketitle
\bigskip



\section{question}
Consider a boolean gate(D) where output $Y$ is related to the inputs $A$ and $B$ as, $Y = A + \overline{B}$, where $+$ denotes logical OR operation. The Boolean inputs '$0$' and '$1$' are also available separately. Using instances of only D gates and inputs '$0$' and '$1$' (Select the correct options)
\begin{enumerate}
    \item [(A)] NAND logic can be implemented
    \item [(B)] OR logic cannot be implemented 
    \item [(C)] NOR logic can be implemented
    \item [(D)] AND logic cannot be implemented
\end{enumerate}

\section{Solution}
If a NAND gate or NOR gate is implemented from the D-Gate, it will be a universal gate.\\
For NOT gate:
\begin{align}
    \overline{A} &= Dgate\brak{0,A}
\end{align}
For NAND gate:
\begin{align}
    \overline{A.B} &= Dgate\brak{Dgate\brak{0,A},B}
\end{align}
Therefore D-gate is a universal gate and all logic gates specified in the options can be implemented making (A),(C) to be the right options.\\
Code for implementation through C onto Vaman-ARM.\\
\url{https://github.com/Gandubs/Digital-Design/tree/master/Assignments/ec'22-49}
 \end{document}
