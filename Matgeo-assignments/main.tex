    %\iffalse
\let\negmedspace\undefined
\let\negthickspace\undefined
\documentclass[journal,12pt,onecolumn]{IEEEtran}
\usepackage{cite}
\usepackage{amsmath,amssymb,amsfonts,amsthm}
\usepackage{algorithmic}
\usepackage{graphicx}
\usepackage{textcomp}
\usepackage{xcolor}
\usepackage{txfonts}
\usepackage{listings}
\usepackage{enumitem}
\usepackage{mathtools}      
\usepackage{gensymb}
\usepackage{comment}
\usepackage[breaklinks=true]{hyperref}
\usepackage{tkz-euclide} 
\usepackage{listings}
\usepackage{gvv}    
\usepackage{enumitem}
\usepackage{amsmath}
\usepackage{amssymb}
\def\inputGnumericTable{}                                 
\usepackage[latin1]{inputenc}                                
\usepackage{color}                                            
\usepackage{array}                                            
\usepackage{longtable}                                       
\usepackage{calc}                                             
\usepackage{multirow}                                         
\usepackage{hhline}                                           
\usepackage{ifthen}                                           
\usepackage{lscape}
\usepackage{tabularx}
\usetikzlibrary{shapes.gates.logic.US, circuits.logic.US}

\newtheorem{theorem}{Theorem}[section]
\newtheorem{problem}{Problem}
\newtheorem{proposition}{Proposition}[section]
\newtheorem{lemma}{Lemma}[section]
\newtheorem{corollary}[theorem]{Corollary}
\newtheorem{example}{Example}[section]
\newtheorem{definition}[problem]{Definition}
\newcommand{\BEQA}{\begin{eqnarray}}
\newcommand{\EEQA}{\end{eqnarray}}
\newcommand{\define}{\stackrel{\triangle}{=}}
\theoremstyle{remark}
\newtheorem{rem}{Remark}
\begin{document}
\bibliographystyle{IEEEtran}
\vspace{3cm}

\title{NCERT}
\author{Ananth $^{}$% <-this % stops a space
}
\maketitle
\bigskip



\subsection{ncert-10th-(6)}
\subsubsection{Question}
The centre of a circle is at (2,0). If one end if a diameter is at (6,0), then the other end is at :
\begin{enumerate}
\item $\brak{0,0}$
\item $\brak{4,0}$
\item $\brak{-2,0}$
\item $\brak{-6,0}$
\end{enumerate}
\subsubsection{Solution}
Let centre of circle be denoted by $\textbf{O}$ and one end of diameter by $\textbf{A}$.Let $\textbf{B}$ be the other end of the diameter.
\begin{align}
\textbf{O} &=
    \begin{pmatrix}
2 \\
0 
\end{pmatrix}\\
\textbf{A} &=
    \begin{pmatrix}
6 \\
0 
\end{pmatrix}
\\
\textbf{O} &= \dfrac{\brak{\textbf{A} + \textbf{B}}}{2}\\
\textbf{B} &= 2\textbf{O} - \textbf{A}\\
&=  2\begin{pmatrix}
2 \\
0 
\end{pmatrix} -  \begin{pmatrix}
6 \\
0 
\end{pmatrix}
=\begin{pmatrix}
   -2\\
   0
\end{pmatrix}
\end{align}
$\implies \text{Option (C) is true}$



\subsection{ncert-10th-12}
\subsubsection{Question}
$AD$ is a median of $\Delta ABC$ with vertices $A\brak{5,-6}$,$B\brak{6,4}$ and $C\brak{0,0}$. Length $AD$ is equal to:
\begin{enumerate}
\item $\sqrt{68}$
\item $2\sqrt{15}$
\item $\sqrt{101}$
\item $10$
\end{enumerate}
\subsubsection{Solution}
As Midpoint of $\textbf{BC}$ is given by $\textbf{D}$
\begin{align}
    \textbf{D} &= \dfrac{\textbf{B} + \textbf{C}}{2}\\
    &=\dfrac{1}{2}\begin{pmatrix}
        6\\
        4
    \end{pmatrix}
    +
   \dfrac{1}{2} \begin{pmatrix}
        0\\
        0
    \end{pmatrix} = \begin{pmatrix}
        3\\
        2
    \end{pmatrix}
\end{align}
Distance between $\textbf{AD}$ is given by it's norm.
\begin{align}
\textbf{A}-\textbf{D} &= \begin{pmatrix}
        5\\
        -6
    \end{pmatrix}-\begin{pmatrix}
        3\\
        2
    \end{pmatrix}=
    \begin{pmatrix}
        2\\
        -8
    \end{pmatrix}
\\    \lvert\lvert \textbf{A}-\textbf{D} \rvert\rvert &\triangleq \sqrt{\brak{\textbf{A}-\textbf{D}}^\top\brak{\textbf{A}-\textbf{D}}} \label{9}\\
    &= \sqrt{\begin{pmatrix}
        2 & -8 
    \end{pmatrix}
    \begin{pmatrix}
        2\\
        -8
    \end{pmatrix}}
    = \sqrt{2^2+8^2} =\sqrt{68} 
\end{align}
$\implies$ Option (A) is true.




\subsection{ncert-10th-14}
\subsubsection{Question}
If the distance between the points $\brak{3,-5}$ and $\brak{x,-5}$ is $15$ units, then the values of $x$ are
\begin{enumerate}
\item $12,-18$
\item $-12,18$
\item $18,5$
\item $-9,-12$
\end{enumerate}
\subsubsection{Solution}
Let $A$ and $B$ denote the points and the distance between them be denoted by norm of $\textbf{A}$-$\textbf{B}$.
\begin{align}
\textbf{A} &=
    \begin{pmatrix}
3 \\
-5 
\end{pmatrix}\\
\textbf{B} &=
    \begin{pmatrix}
x \\
-5 
\end{pmatrix}\\
\textbf{A}-\textbf{B} &= \begin{pmatrix}
        3-x\\
        -5 -\brak{-5}
    \end{pmatrix} = \begin{pmatrix}
        3-x\\
        0
    \end{pmatrix}\\
d &= \lvert\lvert \textbf{A}-\textbf{B} \rvert\rvert \triangleq \sqrt{\brak{A-B}^\top\brak{A-B}}\\
&= \sqrt{\begin{pmatrix}
        3-x & 0
    \end{pmatrix}
\begin{pmatrix}
        3-x\\
        0
    \end{pmatrix}}
     = \sqrt{\brak{3-x}^2}\\
  15 &= \pm\brak{3-x}\\
 \implies  x &= -12,18
\end{align}
$\implies$ Option (B) is true.
\subsection{ncert-10th-24}
\subsubsection{question}
Solve the following system of linear equations algebraically:\\
$2x+5y=-4$;$4x-3y$=$5$
\subsubsection{solution}
The above system of equations can be written as:
\begin{align}
    \begin{pmatrix}
        2 & 5\\ 
        4 & -3
    \end{pmatrix}
    \begin{pmatrix}
        x\\
        y
    \end{pmatrix} &= \begin{pmatrix}
        -4\\
        5
    \end{pmatrix} \label{8}
    \end{align}
    Writing the augmented matrix for using Gauss elimination
    \begin{align}
        \begin{amatrix}{2}
   2 & 5 & -4 \\  4 & -3 & 5
 \end{amatrix} \xleftarrow{R_2 \to R_2-2R_1}
  \begin{pmatrix}
   2 & 5 & -4 \\  0 & -13 & 13
 \end{pmatrix} \xleftarrow{R_1 \to \frac{13}{5}R_1+R_2}
 \begin{pmatrix}
   \frac{26}{5} & 0 & \frac{13}{5} \\  0 & -13 & 13
 \end{pmatrix}  \label{14}
 \end{align}
 \begin{align}
     \implies \begin{pmatrix}
        x\\
        y
    \end{pmatrix} =
    \begin{pmatrix}
        \dfrac{1}{2}\\
        -1
    \end{pmatrix}
    \end{align}
    \subsection{ncert-10th-28}
\subsubsection{question}
The sum of the digits of a $2$-digit number is 14. The number obtained by interchanging     its digits exceeds the given number by 18. Find the number.
\subsubsection{solution}
Let the digits of the number be $x_1$(tens) and $x_2$(units).Given
\begin{align}
    x_1+x_2 &= 14 \label{1}\\
    10x_2+x_1 &= 18 + 10x_1+x_2\\
    \implies x_1-x_2 &= -2\label{2}
\end{align}
Solving the equations $\eqref{1}$,$\eqref{2}$ in their matrix forms
\begin{align}
    \begin{pmatrix}
        1 & 1\\
        -1 & 1
    \end{pmatrix}
    \begin{pmatrix}
        x_1\\
        x_2
    \end{pmatrix} &= \begin{pmatrix}
        14\\
        -2
    \end{pmatrix}   \\
    \end{align}
    Let:
    \begin{align}
        \textbf{A}=\dfrac{1}{\sqrt{2}}\begin{pmatrix}
        1 & 1\\
        -1 & 1
    \end{pmatrix}\\
    AA^\top = \textbf{I}    
    \end{align}
    \textbf{A} is an orthogonal matrix.
    \begin{align}
            \begin{pmatrix}
        1 & 1\\
        -1 & 1
    \end{pmatrix}
    \begin{pmatrix}
        1 & 1\\
        -1 & 1
    \end{pmatrix}
    \begin{pmatrix}
        x_1\\
        x_2
    \end{pmatrix} &= 
    \begin{pmatrix}
        1 & 1\\
        -1 & 1
    \end{pmatrix}
    \begin{pmatrix}
        14\\
        -2
    \end{pmatrix} \\
    2\textbf{I}\textbf{x} &= \begin{pmatrix}
        12\\
        16
    \end{pmatrix}\\
    \implies \textbf{x} &= \begin{pmatrix}
        6\\
        8
    \end{pmatrix}
\end{align}
\subsection{ncert-10th-30(A)}
\subsubsection{question}
Find the ratio in which the point $\brak{\dfrac{8}{5},y}$ divides the line segment joining the points $\brak{1,2}$ and $\brak{2,3}$.Also, find the value of $y$.  
\subsubsection{solution}
Let the points be denoted by $\textbf{A}$,$\textbf{B}$ and $\textbf{C}$ respectively.
\begin{align}
  A &= \begin{pmatrix}
        1\\
        2
    \end{pmatrix}  \\
    B &= \begin{pmatrix}
        2\\
        3
    \end{pmatrix}\\
  C &= \begin{pmatrix}
        \frac{8}{5}\\
        y
    \end{pmatrix}
    \end{align}
    For Collinearity:
    \begin{align}
          \text{rank}\begin{pmatrix}
        1 & 1 & 1\\
        A & B & C
    \end{pmatrix} &=2 \label{10}\\
    \begin{pmatrix}
        1 & 1 & 1\\
        1 & 2 & 8/5\\
        2 & 3 & y
    \end{pmatrix} &\xleftrightarrow{}
    \begin{pmatrix}
        1 & 1 & 1\\
        0 & \brak{2-1} & \brak{\frac{8}{5}-1}\\
        0 & \brak{3-2} & \brak{y-3}
    \end{pmatrix}\xleftrightarrow{}
    \begin{pmatrix}
        1 & 1 & 1\\
        0 & 1 & \frac{3}{5}\\
        0 & 1 & y-3
    \end{pmatrix}\xleftrightarrow{R_3\rightarrow R_3-R_2}
    \begin{pmatrix}
        1 & 1 & 1\\
        0 & 1 & \frac{3}{5}\\
        0 & 0 & y-\frac{18}{5}
    \end{pmatrix}\\
    \implies y &= \frac{18}{5}
\end{align}

\subsection{ncert-10th-30(B)}
\subsubsection{question}
$ABCD$ is a rectangle formed by the points $A\brak{-1,-1}$,$B\brak{-1,6}$,$C\brak{3,6}$ and $D\brak{3,-1}$. $P$,$Q$,$R$ and $S$ are mid-points of sides $AB$,$BC$,$CD$ and $DA$ respectively. Show that the diagonal of the quadrilateral $PQRS$ bisect each other. 
\subsubsection{solution}
\begin{align}
    A &= \begin{pmatrix}
        -1\\
        -1
    \end{pmatrix},
    B = \begin{pmatrix}
        -1\\
        6
    \end{pmatrix},
    C = \begin{pmatrix}
        3\\
        6
    \end{pmatrix},
    D = \begin{pmatrix}
        3\\
        -1
    \end{pmatrix}\\
    \textbf{P}&=\dfrac{\textbf{A}+\textbf{B}}{2}\\
        \textbf{Q}&=\dfrac{\textbf{B}+\textbf{C}}{2}\\
    \textbf{R}&=\dfrac{\textbf{C}+\textbf{D}}{2}\\
    \textbf{S}&=\dfrac{\textbf{D}+\textbf{A}}{2}\\
    \end{align}
Let $\textbf{O}_1$ and $\textbf{O}_2$ be the midpoints of $PR$ and $QS$ respectively
\begin{align}
    \textbf{O}_1 = \dfrac{\textbf{P}+\textbf{R}}{2}=\dfrac{\textbf{A}+\textbf{B}+\textbf{C}+\textbf{D}}{4}\\
      \textbf{O}_2 = \dfrac{\textbf{Q}+\textbf{S}}{2}=\dfrac{\textbf{A}+\textbf{B}+\textbf{C}+\textbf{D}}{4}
\end{align}
Since the midpoints of the diagonals coincide, the diagonals bisect each other. 



\subsection{ncert-10th-34(A)}
\subsubsection{question}
The sum of first and eight terms of an A.P is $32$ and their product is $60$. Find the first term and common difference of the A.P. Hence, also find the sum of its first $20$ terms.  
\subsubsection{solution}
Let the first and eighth terms be $x\brak{0}$ and $x\brak{7}$ respectively,given:
\begin{align}
x\brak{0}+x\brak{7} &= 32\\
x\brak{0} &= 32 - x\brak{7}\label{3} \\
x\brak{0}x\brak{7} &= 60 \label{4}
\end{align}
From $\eqref{3}$ and $\eqref{4}$
\begin{align}
    x\brak{7}\brak{32-x\brak{7}} &= 60
\end{align}
The roots are $\brak{30,2}$, therefore ,if $x\brak{7}=30$ then $x\brak{0}=2$ and if $x\brak{7}=2$ then $x_0=30$\\
Now
\begin{align}
    x\brak{n} &= \brak{x\brak{0} + nd}u_{\brak{n}}\label{5}
    \end{align}
    Where $d$ is the common difference of the A.P and $u_n$ is the unit step function.\\($u_{\brak{n}}=0 \forall n<0$,$u_{\brak{n}}=1 \forall n\geq0$)
    \begin{align}
        \implies     x\brak{7} &= \brak{x\brak{0} + 7d}\\
    \implies 7d &= \pm 28 \implies d = \pm 4
\end{align}
Therefore the A.P is $2,6,10...$ or $30,26,22...$.\\
Considering the former for calculations and taking Z-Transform of $\eqref{5}$ for sum.\\
Since
\begin{align}
X(z) &= \sum_{n=-\infty}^{\infty} x(n)z^{-n} \label{6}
\end{align}
Let $y\brak{n}$ denote the sum, let:
\begin{align}
    y\brak{n} &= x\brak{n} * h\brak{n}\\
    &= \sum_{k=-\infty}^{\infty} x(k)h(n-k)
\end{align}
Replace $h\brak{n}$ with $u_{\brak{n}}$.
\begin{align}
    y\brak{n} &= \sum_{k=0}^{n} x(k)u_{\brak{n-k}}\\
    &=x(0)u_{\brak{n}} + x(1)u_{\brak{n-1}} + ..... x(n)u_{\brak{0}}
\end{align}
This denotes the sum of terms $x(0),x(1)....x(n)$ i.e. first $n+1$ terms. From \eqref{6}\\
\begin{align}
    u_{\brak{n}} &\xrightarrow{\mathcal{Z}} \frac{1}{\brak{1-z^{-1}}} \label{7}\\
nu_{\brak{n}} &\xrightarrow{\mathcal{Z}} \frac{z^{-1}}{\brak{1-z^{-1}}^{2}}\\
  \implies  X(z)&= \frac{2}{\brak{1-z^{-1}}} + \frac{4z^{-1}}{\brak{1-z^{-1}}^{2}} \label{eq:ee25-4}
,\quad \abs {z}>\abs{1} 
\end{align}
Now as convolution in the time domain corresponds to multiplication in the frequency domain and \eqref{eq:ee25-4} and \eqref{7}.
\begin{align}
    Y\brak{z} &= X\brak{z} * H\brak{z}\\
 &= \brak{\frac{2}{\brak{1-z^{-1}}} +
\frac{4z^{-1}}{\brak{1-z^{-1}}^{2}}}\brak{\frac{1}{\brak{1-z^{-1}}}}
,\quad \abs {z}>\abs{1}     
\end{align}
Using normal inversion for inverse Z-transform:
\begin{align}
 Y(z) &= \frac{2}{\brak{1-z^{-1}}^2} +
\frac{4z^{-1}}{\brak{1-z^{-1}}^{3}}
,\quad \abs {z}>\abs{1}\\
   &= \frac{8z^{-1}}{1-z^{-1}} + \frac{10z^{-2}}{\brak{1-z^{-1}}^2} + \frac{4z^{-3}}{\brak{1-z^{-1}}^{3}} + 2\label{final}
\end{align}
For proceeding forwards here are some important generalizations.\\
Shifting property
\begin{align}
x(n-k) \leftrightarrow z^{-k} X(z) \label{shift}
\end{align}
Differentiation property
\begin{align}
nx(n) \leftrightarrow -zX^{\prime}(z) \label{diff}
\end{align}
From \eqref{7} and \eqref{shift}
\begin{align}
u_{\brak{n-1}} &\xrightarrow{\mathcal{Z}} \frac{z^{-1}}{1-z^{-1}}
\end{align}
From $\eqref{7}$ and $\eqref{diff}$
\begin{align}
nu_{\brak{n}} &\xrightarrow{\mathcal{Z}} -z\frac{d}{dz}\brak{\frac{1}{1-z^{-1}}}\\
nu_{\brak{n}}    &\xrightarrow{\mathcal{Z}} \frac{z^{-1}}{\brak{1-z^{-1}}^{2}}
\end{align}
From $\eqref{shift}$
\begin{align}
(n-1)u_{\brak{n-1}}    &\xrightarrow{\mathcal{Z}} z^{-1}\frac{z^{-1}}{\brak{1-z^{-1}}^{2}}\\
(n-1)u_{\brak{n-1}}    &\xrightarrow{\mathcal{Z}} \frac{z^{-2}}{\brak{1-z^{-1}}^{2}}
\end{align}
Now, using $\eqref{diff}$ and writing the corresponding L.H.S
\begin{align}   
(n)(n-1)u_{\brak{n-1}}    &\xrightarrow{\mathcal{Z}} \frac{2z^{-2}}{\brak{1-z^{-1}}^3} \\
\end{align}
Using $\eqref{shift}$
\begin{align}
    \frac{(n-1)(n-2)u_{\brak{n-2}}}{2}    &\xrightarrow{\mathcal{Z}} \frac{z^{-3}}{\brak{1-z^{-1}}^3} \\
\end{align}
The inverse-Z of a constant will be $\delta(n)$, so it is ruled out.
Plugging these values in $\eqref{final}$ we get
\begin{align}
   y(n) &= 8u_{\brak{n-1}} + 10(n-1)u_{\brak{n-1}} + 4\frac{(n-1)(n-2)u_{\brak{n-2}}}{2}
\end{align}
Putting $n=19$
\begin{align}
      y(19) &= 2(19+1)^2 = 800
\end{align}
Using contour integration for inverse Z-transform
\begin{align}
    y(19)&=\frac{1}{2\pi j}\oint_{C}Y(z) \;z^{18} \;dz\\  
 &=\frac{1}{2\pi j}\oint_{C}\brak{2z^{20}\brak{z-1}^{-2}+
       4z^{20}\brak{z-1}^{-3}} \;dz\\
       R&=\frac{1}{\brak {m-1}!}\lim\limits_{z\to a}\frac{d^{m-1}}{dz^{m-1}}\brak {{(z-a)}^{m}f\brak z}\label{eq:6}  
\end{align}
For $R_1$ , $m=2$ , where $m$ corresponds to number of repeated poles .
\begin{align}
    R_1 &=\frac{1}{\brak {1}!}\lim\limits_{z\to 1}\frac{d}{dz}\brak {{(z-1)}^{2}2z^{20}\brak{z-1}^{-2}}   \\
    &=2\lim\limits_{z\to 1}\frac{d}{dz}(z^{20})   \\
    &= 40
    \end{align}
    For $R_2$ , $m=3$ 
    \begin{align}
    R_2 &=\frac{1}{\brak {2}!}\lim\limits_{z\to 1}\frac{d^{2}}{dz^{2}}\brak {{(z-1)}^{3}4z^{20}\brak{z-1}^{-3}}   \\
    &=\brak2\lim\limits_{z\to 1}\frac{d^2}{dz^2}(z^{20})   \\
    &= 760\\
    R_1 + R_2 &= 800\\
    \implies  y{(19)} &= 800
\end{align}
Similarly, the sum for the A.P. $30,26,22...$ can be found by the same procedure.
\subsection{ncert-10th-34(B)}
\subsubsection{question}
In an A.P. of $40$ terms, the sum of first $9$ terms is $153$ and the sum of last $6$ terms is $687$. Determine the first term and the common difference of the A.P. Also find the sum of all the terms of the A.P. 
\subsubsection{solution}
Given:
\begin{align}
y(8) &= 153 \label{17}\\
y(39)-y(34) &= 687 \label{18}
\end{align}
Now, let the first term be $x(0)$ and common difference be $d$. From $\eqref{6}$ and $\eqref{5}$
\begin{align}
    X(z) &= \dfrac{x(0)}{1-z^{-1}} + \dfrac{dz^{-1}}{\brak{1-z^{-1}}^2} ,\quad \abs {z}>\abs{1} 
\end{align}
For finding the sum (Assuming $h(n)=u_{\brak{n}}$
\begin{align}
    y\brak{n} &= x\brak{n} * h\brak{n}\\
Y\brak{z} &= X\brak{z} * H\brak{z}\\
&= \brak{\frac{x(0)}{\brak{1-z^{-1}}} +
\frac{dz^{-1}}{\brak{1-z^{-1}}^{2}}}\brak{\frac{1}{\brak{1-z^{-1}}}}
,\quad \abs {z}>\abs{1}     
\end{align}
Using normal inversion for inverse Z-transform:
\begin{align}
   hi
\end{align}







Using contour integration for inverse Z-transform
\begin{align}
    y(n)&=\frac{1}{2\pi j}\oint_{C}Y(z) \;z^{n-1} \;dz\\  
 &=\frac{1}{2\pi j}\oint_{C}\brak{x(0)z^{n+1}\brak{z-1}^{-2}+
       dz^{20}\brak{z-1}^{-3}} \;dz\\
       R&=\frac{1}{\brak {m-1}!}\lim\limits_{z\to a}\frac{d^{m-1}}{dz^{m-1}}\brak {{(z-a)}^{m}f\brak z}\label{eq:6}  
\end{align}
For $R_1$ , $m=2$ , where $m$ corresponds to number of repeated poles .
\begin{align}
    R_1 &=\frac{1}{\brak {1}!}\lim\limits_{z\to 1}\frac{d}{dz}\brak {{(z-1)}^{2}x(0)z^{n+1}\brak{z-1}^{-2}}   \\
    &=x(0)\lim\limits_{z\to 1}\frac{d}{dz}(z^{n+1})   \\
    &= \brak{n+1}x(0)
    \end{align}
    For $R_2$ , $m=3$ 
    \begin{align}
    R_2 &=\frac{1}{\brak {2}!}\lim\limits_{z\to 1}\frac{d^{2}}{dz^{2}}\brak {{(z-1)}^{3}dz^{n+1}\brak{z-1}^{-3}}   \\
        &=\brak{\frac{d}{2}}\lim\limits_{z\to 1}\frac{d^2}{dz^2}(z^{n+1})   \\
    &= \brak{\frac{d}{2}}\brak{n}\brak{n+1}\\
    R_1 + R_2 &= \brak{\frac{n+1}{2}}\brak{x(0) + \frac{nd}{2}} \label{19}\\
\end{align}
Now use $\eqref{17}$ and $\eqref{18}$ to solve for $x(0)$ and $d$ and put in $\eqref{19}$ for the sum of $40$ terms.


\subsection{ncert-12th-14}
\subsubsection{question}
If $\overrightarrow{a} = 2\hat{i} - \hat{j} + \hat{k}$ and  $\overrightarrow{b} = \hat{i} + \hat{j} - \hat{k}$, then $\overrightarrow{a}$ and $\overrightarrow{b}$ are:
    \begin{enumerate}
\item Collinear vectors which are not parallel
\item Parallel vectors
\item Perpendicular vectors
\item Unit vectors
\end{enumerate}
\subsubsection{solution}
Let
\begin{align}
    \textbf{a} &= \begin{pmatrix}
        2\\
        -1\\
        1
    \end{pmatrix} , 
    \textbf{b}=\begin{pmatrix}
        1\\
        1\\
        -1
    \end{pmatrix}
    \end{align}
    Applying concept of rank from $\eqref{10}$
    \begin{align}
                \\  \text{rank}\begin{pmatrix}
        1 & 1 & -1\\
        2 & -1 & 1
    \end{pmatrix} &=2 \neq 1 ,
    \text{Not parallel}
    \end{align}
    Applying condition for perpendicularity:
    \begin{align}
            \textbf{a}^{\top}\textbf{b} = \begin{pmatrix}
        2 &-1 &1
    \end{pmatrix}\begin{pmatrix}
        1\\
        1\\
        -1
    \end{pmatrix} &= 0
    \implies \textbf{a} \perp \textbf{b}
\end{align}
$\implies$ Option (C) is true.




\subsection{ncert-12th-14}
\subsubsection{question}
If $\alpha$,$\beta$ and $\gamma$ are the angles which a line makes with positive directions of $x$,$y$ and $z$ axes respectively, then which of the following are \textbf{not} true?
    \begin{enumerate}
\item $\cos^2{\alpha} + \cos^2{\beta} + \cos^2{\gamma} = 1$
\item $\sin^2{\alpha} + \sin^2{\beta} + \sin^2{\gamma} = 2$
\item $\cos{2\alpha} + \cos{2\beta} + \cos{2\gamma} =-1$
\item $\cos{\alpha} + \cos{\beta} + \cos{\gamma} = 1$
\end{enumerate}
\subsubsection{solution}
Let $\textbf{m}$ represent the direction vector of the line:
\begin{align}
    \textbf{m} &= \begin{pmatrix}
        \cos{\alpha} \\
        \cos{\beta}\\
        \cos{\gamma}
    \end{pmatrix}
\end{align}
For option (A)
\begin{align}
\textbf{m}\textbf{m}^{\top} &= 1
\end{align}
The following is the normal equation of a line, where $\textbf{n}$ represents the normal vector.
\begin{align}
    \textbf{n}^{\top}\textbf{x} &= c
\end{align}
From $\eqref{8}$
\begin{align}
    2x + 5y &= -4\\
    2x &= -4 -5y\\
    \begin{pmatrix}
        x\\
        y
    \end{pmatrix} &= \begin{pmatrix}
        -2\\
        0
    \end{pmatrix} + y\begin{pmatrix}
        -\frac{5}{2}\\
        1
    \end{pmatrix}\\
    \textbf{x} &= \begin{pmatrix}
        -2\\
        0
    \end{pmatrix} -\frac{5y}{2}\begin{pmatrix}
        1\\
        -\frac{2}{5}
    \end{pmatrix}\\
    &= \textbf{A} + K\textbf{m} \label{11}
\end{align}
The general equation written above is of the parametric form.\\
Similarly apply the concept of finding $\textbf{m}$ for this question and solve.



\subsection{ncert-12th-21}
\subsubsection{question}
$\overrightarrow{a}$,$\overrightarrow{b}$ and $\overrightarrow{c}$ are three mutually perpendicular unit vectors. If $\theta$ is the angle between $\overrightarrow{a}$ and $\brak{\overrightarrow{2a}+\overrightarrow{3b}+\overrightarrow{6c}}$, find the value of $\cos{\theta}$. 
\subsubsection{solution}
Given:
\begin{align}
    \textbf{a}^{\top}\textbf{b} &=  \textbf{b}^{\top}\textbf{c} =  \textbf{c}^{\top}\textbf{a} = 0\\
    \lvert \lvert \textbf{a} \rvert \rvert 
&= \lvert \lvert \textbf{b} \rvert \rvert = \lvert \lvert \textbf{c} \rvert \rvert = 1\\
\cos{\theta} &= \frac{\textbf{a}^{\top}\brak{2\textbf{a}+3\textbf{b}+6\textbf{c}}}{\lvert \lvert \textbf{a} \rvert \rvert \lvert \lvert 2\textbf{a} + 3\textbf{b} + 6\textbf{c} \rvert \rvert}
\end{align}
Now,
\begin{align}
    \textbf{a}^{\top}\brak{2\textbf{a}+3\textbf{b}+6\textbf{c}} &= 2\textbf{a}^{\top}\textbf{a} + 3\textbf{a}^{\top}\textbf{b} + 6\textbf{a}^{\top}\textbf{c} = 2 + 0 + 0 = 2\\
    \lvert \lvert \textbf{a} \rvert \rvert \lvert \lvert 2\textbf{a} + 3\textbf{b} + 6\textbf{c} \rvert \rvert &= \lvert \lvert 2\textbf{a} + 3\textbf{b} + 6\textbf{c} \rvert \rvert\\
    \end{align}
From $\eqref{9}$ norm definition:
    \begin{align}
        \brak{\lvert \lvert 2\textbf{a} + 3\textbf{b} + 6\textbf{c} \rvert \rvert}^2 &= \lvert \lvert 4\textbf{a}^2 \rvert \rvert + \lvert \lvert 9\textbf{b}^2 \rvert \rvert + \lvert \lvert 36\textbf{c}^2  \rvert \rvert + \lvert \lvert 12\textbf{a}^{\top}\textbf{b} \rvert \rvert + \lvert \lvert 36\textbf{b}^{\top}\textbf{c} \rvert \rvert + \lvert \lvert 24\textbf{c}^{\top}\textbf{a} \rvert \rvert.
    = 49\\
    \implies \lvert \lvert 2\textbf{a} + 3\textbf{b} + 6\textbf{c} \rvert \rvert &= +7\\
    \implies \cos{\theta} &= \dfrac{2}{7}
\end{align}

\subsection{ncert-12th-25}
\subsubsection{question}
Find the position vector of point $\textbf{C}$ which divides the line segment joining points $\textbf{A}$ and $\textbf{B}$ having position vectors $\hat{i} + 2\hat{j} - \hat{k}$ and $-\hat{i} + \hat{j} + \hat{k}$ respectively in the ratio $4:1$ externally. Further, find $\lvert \overrightarrow{AB}\rvert : \lvert \overrightarrow{BC} \rvert$. 
\subsubsection{solution}
We know that:
\begin{align}
\textbf{C} = \frac{4\textbf{B}-\textbf{A}}{4-1}
\end{align}
Simplify the above for \textbf{C} (Given in $10^{th}$ NCERT).


\subsection{ncert-12th-32(a)}
\subsubsection{question}
Find the equation of the line passing through the point of intersection of the lines $\frac{x}{1} = \frac{y-1}{2} = \frac{z-2}{3}$ and $\frac{x-1}{0} = \frac{y}{-3} = \frac{z-7}{2}$ and perpendicular to these given lines. 
\subsubsection{solution}
Let the given lines be denoted by $\textbf{x}_1$ and $\textbf{x}_2$ respectively. From $\eqref{11}$:
\begin{align}
    \textbf{x}_1 &= \begin{pmatrix}
        0\\
        1\\
        2
    \end{pmatrix} + k_1\begin{pmatrix}
        1\\
        2\\
        3
    \end{pmatrix} = \textbf{A} + k_1\textbf{m}_1 \label{12} \\
    \textbf{x}_2 &= \begin{pmatrix}
        1\\
        0\\
        7
    \end{pmatrix} + k_2\begin{pmatrix}
        0\\
        -3\\
        2
    \end{pmatrix} = \textbf{B} + k_2\textbf{m}_2 \label{13}
\end{align}
Equation of a line passing through point of intersection of two lines is given by $\textbf{x}_1 + \lambda\textbf{x}_2$, let the direction vector of the line be denoted by $\textbf{m}$
\begin{align}
    \brak{\textbf{A} + k_1\textbf{m}_1} + \lambda\brak{\textbf{B} + k_2\textbf{m}_2} &= 0\\
\end{align}
The two equations required to solve for the direction of line are 
\begin{align}
\textbf{m}_1\textbf{m}^\top &= 0\\
\textbf{m}_2\textbf{m}^\top &= 0\\
\implies \brak{\textbf{m}_1\textbf{m}_2}^{\top}\textbf{m} &= 0
\end{align}
\begin{align}
    \textbf{m}\begin{pmatrix}
        1 & 2 & 3\\
        0 & -3 & 2   \\
    \end{pmatrix} \xleftarrow{R_1 \to 2R_2 + 3R_1} &\begin{pmatrix}
        3 & 0 & 13\\
        0 & -3 & 2   \\
    \end{pmatrix}\textbf{m} = 0\\
    \begin{pmatrix}
        3 & 0 & 13\\
        0 & -3 & 2   \\
    \end{pmatrix}\begin{pmatrix}
        m_x\\
        m_y\\
        m_z
    \end{pmatrix} &= 0\\
    \implies \textbf{m} &= \begin{pmatrix}
        \frac{-13}{3}\\
        \frac{2}{3}\\
        1
    \end{pmatrix}
\end{align}
Let the unknown line in its parametric form be denoted as follows from $\eqref{11}$.
\begin{align}
    \textbf{x}_3 = \textbf{C} + k_3\textbf{m}
\end{align}
For finding the point of intersection of the lines($\textbf{C}$), equating $\eqref{12}$ and $\eqref{13}$
\begin{align}
    \textbf{A} + k_1\textbf{m}_1 = \textbf{B} + k_2\textbf{m}_2\\
    \begin{pmatrix}
        \textbf{m}_1 & \textbf{m}_2
    \end{pmatrix}\begin{pmatrix}
        k_1 \\
        -k_2
    \end{pmatrix} = \textbf{B}-\textbf{A}
\end{align}
From the above $k_1$ and $k_2$ can be found by gauss elimination given in $\eqref{14}$ and thus $\textbf{C}$.

\subsection{ncert-12th-32(b)}
\subsubsection{question}
Two vertices of the parallelogram $\textbf{ABCD}$ are given as $\textbf{A}\brak{-1,2,1}$ and $\textbf{B}\brak{1,-2,5}$. If the equation if the line passing through $\textbf{C}$ and $\textbf{D}$ is $\frac{x-4}{1} = \frac{y+7}{-2} = \frac{z-8}{2}$, then find the distance between the sides $\textbf{AB}$ and $\textbf{CD}$. Hence, find the area of parallelogram $\textbf{ABCD}$
\subsubsection{solution}
Given:
\begin{align}
    \textbf{A} &= \begin{pmatrix}
        -1\\
        2\\
        1
    \end{pmatrix},
    \textbf{B} = \begin{pmatrix}
        1\\
        -2\\
        5
    \end{pmatrix}\\
    \textbf{CD}:\textbf{x} &= \begin{pmatrix}
        4\\
        -7\\
        8
    \end{pmatrix}
    +k\begin{pmatrix}
        1\\
        -2\\
        2
    \end{pmatrix}
\end{align}
Let the corresponding foot of perpendicular of $\textbf{A}$ on line $\textbf{CD}$ be $\textbf{P}$, the direction vector of line $\textbf{CD}$ be $\textbf{m}_1$ . Therefore the required equations are:
\begin{align}
    \brak{\textbf{A}-\textbf{P}}^{\top} \textbf{m}_1 &= 0 \label{15}\\
    \textbf{P} &= \textbf{Q} + k\textbf{m}_1 \label{16}
\end{align}
Where $\textbf{Q}$ is the given vector lying on $\textbf{CD}$.\\
From $\eqref{15}$ and $\eqref{16}$, $k$ can be found out, which when substituted in $\eqref{16}$ gives $\textbf{P}$. Distance between $\textbf{A}$ and $\textbf{P}$ is given by $\eqref{9}$.
\end{document}
